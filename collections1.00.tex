
\documentclass[color=green,mathpazo,titlestyle=hang]{elegantbook}

\author{Huyi Chen}
\email{hooyuser@outlook.com}
\zhtitle{数学}
\zhend{收集箱}

\entitle{Collections}
\enend{Of Math}
\version{1.00}
\myquote{If people do not believe that mathematics is simple,\\ 
	it is only because they do not realize how complicated life is.}
\logo{box.pdf}
\cover{grass.pdf}
  
%green color
   \definecolor{main1}{RGB}{0,120,2}
   \definecolor{seco1}{RGB}{230,90,7}
   \definecolor{thid1}{RGB}{0,160,152}
%cyan color
   \definecolor{main2}{RGB}{0,175,152}
   \definecolor{seco2}{RGB}{239,126,30}
   \definecolor{thid2}{RGB}{120,8,13}
%blue color
   \definecolor{main3}{RGB}{20,50,104}
   \definecolor{seco3}{RGB}{180,50,131}
   \definecolor{thid3}{RGB}{7,127,128}

\usepackage{makecell}
\usepackage{lipsum}
\usepackage{texnames} 
\usepackage{amssymb}
\usepackage[all]{xy}


\begin{document}
\maketitle
\tableofcontents
\mainmatter

\chapter{数论}

\section{中国剩余定理}

\subsection{历史背景}
《孙子算经》是中国南北朝时期(公元5世纪)的数学著作\cite{szsj}. 其卷下第二十六题,叫做“物不知数”问题,原文如下:\\ 

\emph{有物不知其数,三三数之剩二,五五数之剩三,七七数之剩二。问物几何?}\\ 


翻译成白话文,即一个整数除以三余二,除以五余三,除以七余二,求这个整数.《孙子算经》中首次提到了同余方程组问题,并给出了以上具体问题的解法,因此在一些中文数学文献中,中国剩余定理也会被称为孙子定理.

宋朝数学家秦九韶于1247年《数书九章》卷一、二《大衍类》对“物不知数”问题做出了完整系统的解答。明朝数学家程大位将解法编成易于上口的《孙子歌诀》\cite{szgj}:\\ 

\emph{三人同行七十希,五树梅花廿一支,七子团圆正半月,除百零五使得知.}\\

这个歌诀给出了模数为3、5、7时候的同余方程的秦九韶解法。意思是:将除以3得到的余数乘以70,将除以5得到的余数乘以21,将除以7得到的余数乘以15,全部加起来后除以105,得到的余数就是答案。比如说在以上的物不知数问题里面,使用以上的方法计算就得到
\[ 70\times 2+21\times 3+15\times 2=233=2\times 105+23.\] 

因此按歌诀求出的结果就是23.



\subsection{定理陈述}
  中国剩余定理有三种常见的表述方式,将在下面一一给出.
  第一种是以余数的形式.我们首先引入带余除法的概念.
  
  
\begin{newprop}[带余除法]
  设$a\in \mathbb{Z}$,$b\in \mathbb{N^+}$.	一定存在唯一的整数对$(q,r)$,使$a=bq+r$,且$0\leq r< b$.其中$b=\left[\dfrac{a}{b}\right]$称作$a$除以$b$的不完全商,$r=a-b\left[\dfrac{a}{b}\right]$称作$a$除以$b$的余数,$r$也常常被记作$a\bmod b$.
\end{newprop} 


\newpage

  命题的证明是平凡的,这里略去.在下面定理的叙述中,我们总是假定$n_1,n_2,\cdots,n_k$是大于1的整数,而$n_i$常常称作模. 同时,我们记$N=n_1n_2\cdots n_k$为所有模的积. 现在给出中国剩余定理的第一种表述.
  
  
  \begin{newthem}[中国剩余定理I]
  	如果$n_i$两两互素,且整数$r_i$满足$0\leq r< n_i$,则存在唯一满足$0\leq x <N$的整数$x$,使得对每一个$i(1\leq i \leq k)$,都有$x\bmod n_i=r_i$.
  \end{newthem}

  上述表述是《孙子算经》中具体问题的一般化描述,但直接处理余数往往并不方便. 若引入同余记号,这个问题实际上就变成如何去求解一个一元一次同余方程组,这也是中国剩余定理的第二种表述,而它与第一种描述是完全等价的.
  
  \begin{newthem}[中国剩余定理II]
  	如果$n_1,n_2,\cdots,n_k$两两互素,且$r_1,r_2,\cdots,r_k \in \mathbb{Z}$,则同余方程组
  
  	\begin{equation*}
  	\left\{                          
  	\begin{aligned}
  	x&\equiv r_1\pmod{n_1}\\
  	x&\equiv r_2\pmod{n_2}\\
  	 &\qquad \vdots\\ 
  	x&\equiv r_k\pmod{n_k}\\
  	\end{aligned}
  	\right.
  	\end{equation*}
  有无穷多解,且任意两个解模$N$同余.	
  
  \end{newthem}


  \begin{newproof}
	先证存在性. 记除了$n_i$外所有模的乘积为$N_i=\dfrac{N}{n_i}$.因为$n_i$两两互素,故$N_i$也与$n_i$互素.由B\'{e}zout等式,存在整数$M_i,m_i$,使得
	\[
	M_iN_i+m_in_i=1.
	\]
	因此
	\[
	N_iM_i\equiv 1\pmod{n_i}.
	\]
	记$M_i=N_i^{-1}$为$N_i$的数论倒数,则x可以构造为$\sum\limits_{i=1}^{k}r_iN_i N_i^{-1}$.事实上,只要注意到$j\ne i$时$n_i|N_j$,于是有
	\[
	x\equiv \sum_{i=1}^{k}r_iN_{i} N_i^{-1}\equiv r_iN_i N_i^{-1}\equiv r_i\pmod{n_i}.
	\]
	再证唯一性.若$y$也是一个解,则$x\equiv y\equiv r_i\pmod{n_i}$. 又因为$n_i$两两互素,由算术基本定理知$x\equiv y\pmod{N}$.
	综上可得,同余方程组的通解是
	\[
		x=\sum_{i=1}^{k}r_iN_i N_i^{-1}+mN \quad (m\in\mathbb{Z}).
	\]
  \end{newproof}


  因为模$n_i$的剩余类构成一个环$\mathbb{Z}_{n_i}=\mathbb{Z}/n_i\mathbb{Z}$,运用抽象代数的语言,中国剩余定理可以描述成一个环同构. 在此之前,我们有必要先明确环的直积的定义.



\begin{newdef}[环的直积]
	给定两个环$(G,\emph{+},\ast)$和$(H,\oplus,\odot)$,它们的直积仍是一个环$(G\times H,+,\cdot)$,其中集合$G\times H=\{(g,h)|g\in G, h\in H\}$是$G$与$H$的笛卡儿积;环上的运算$+,\cdot$定义为
	\begin{itemize}
		\parskip=0pt \itemsep=0pt
		\item $(g_1,h_1)+(g_2,h_2)=(g_1\emph{+}g_2,h_1\oplus h_2)$
		\item $(g_1,h_1)\cdot(g_2,h_2)=(g_1\ast g_2,h_1\odot h_2)$
	\end{itemize}
	在不会引起误解的情形下,环$(G\times H,\emph{+},\ast)$可简记$G\times H$,其乘法单位元为$(1_G,1_H)$. 类似地,对于可数个环,我们也可通过这种分量加法和分量乘法的方式,定义$\{R_i\}_{i\in I}$的直积$\prod\limits_{i\in I}R_i$.
\end{newdef}


  有了环的直积这一个概念后,就可以正式介绍定理的第三种表述了. 这将为我们提供一个更加清晰的视角.

  
  \begin{newthem}[中国剩余定理III]
  	若$n_1n_2\cdots n_k$两两互素,则映射
  	\[
  	\varphi:(x\bmod n_1,x\bmod n_2,\cdots,x\bmod n_k)\mapsto x\bmod N
  	\]
  	确定一个环同构
  	\[
  	\varphi:\mathbb{Z}/n_1\mathbb{Z}\times\mathbb{Z}/n_2\mathbb{Z}\times\cdots\times\mathbb{Z}/n_k\mathbb{Z}\rightarrow\mathbb{Z}/N\mathbb{Z}.
  	\]
  

  \end{newthem}


  \begin{newproof}
	在下面的证明中为了书写简便,对于模$m$剩余类$\bar{x}=x+m\mathbb{Z}=\{x+km|k\in\mathbb{Z}\}\in \mathbb{Z}/m\mathbb{Z}$,我们将它与剩余类的代表元$x$不做区分. 首先证明$\varphi$是一个双射.事实上,如果
	\[
	\varphi(r_1,r_2,\cdots,r_k)=\varphi(r_1^{'},r_2^{'},\cdots,r_k^{'})=R,
	\]
	则有$r_i\equiv r_i^{'}\equiv R \pmod{n_i} $,或$(r_1,r_2,\cdots,r_k)=(r_1^{'},r_2^{'},\cdots,r_k^{'}).$这说明$\varphi$是单射.
	又注意到基数$\left|\mathbb{Z}/n_1\mathbb{Z}\times\mathbb{Z}/n_2\mathbb{Z}\times\cdots\times\mathbb{Z}/n_k\mathbb{Z}\right|=\left|\mathbb{Z}/N\mathbb{Z}\right|=N$,所以$\varphi$一定是双射.
	此外,我们还需验证双射$\varphi$保持运算。根据环的直积的定义,我们有
	
	\begin{equation*}	
	\begin{aligned}
	\varphi(a_1,a_2,\cdots,a_k)+\varphi(b_1,b_2,\cdots,b_k)	
	&=\sum_{i=1}^{k}a_iN_i N_i^{-1}+\sum_{i=1}^{k}b_iN_i N_i^{-1}=\sum_{i=1}^{k}(a_i+b_i)N_i N_i^{-1}\\
	&=\varphi(a_1+b_1,a_2+b_2,\cdots,a_k+b_k)\\
	&=\varphi[(a_1,a_2,\cdots,a_k)+(b_1,b_2,\cdots,b_k)].
	\end{aligned}		
	\end{equation*}
	
	\begin{equation*}	
	\begin{aligned}
	\varphi(a_1,a_2,\cdots,a_k)\cdot\varphi(b_1,b_2,\cdots,b_k)	
	&=\left(\sum_{i=1}^{k}a_iN_i N_i^{-1}\right)\cdot\left(\sum_{i=1}^{k}b_iN_i N_i^{-1}\right)\\
	&=\sum_{i=1}^{k}a_i b_i (N_i N_i^{-1})^2+\sum_{i\ne j}a_iN_i N_i^{-1}b_jN_j N_j^{-1}\\
	&=\sum_{i=1}^{k}a_i b_i N_i N_i^{-1}\\
	&=\varphi(a_1\cdot b_1,a_2\cdot b_2,\cdots,a_k\cdot b_k)\\
	&=\varphi[(a_1,a_2,\cdots,a_k)\cdot(b_1,b_2,\cdots,b_k)].
	\end{aligned}		
	\end{equation*}  
    这就证明了$\varphi$是$\mathbb{Z}/n_1\mathbb{Z}\times\mathbb{Z}/n_2\mathbb{Z}\times\cdots\times\mathbb{Z}/n_k\mathbb{Z}$到$\mathbb{Z}/N\mathbb{Z}$上的环同构.

  \end{newproof}

  从环同构的观点出发,我们可以将定理自然地推广到一般的PID(主理想整环)上,这时$R$模掉极大理想$I$得到的的商环$R/I$代替了原先的剩余类环$\mathbb{Z}/m\mathbb{Z}$. 原因是证明中用到的B\'{e}zout等式在PID上有对应的推广,而算术基本定理(唯一分解定理)在更一般的UFD(唯一分解整环)上也成立. 进一步地,通过定义互素理想,我们还可以将定理推广到任意环上.

\chapter{代数}
\section{基础概念}

\subsection{等价关系}

\begin{newdef}[二元关系]
	设$X,Y$是任意两个集合,其任意子集$\mathcal{R}\in X\times Y$叫做$X$与$Y$之间的一个\textbf{二元关系}.若$X=Y$,则简称为$X$上的一个二元关系.	
\end{newdef}
有序对$(x,y)\in\mathcal{R}$可简记为$x\mathcal{R} y$.

\begin{newdef}[等价关系]
	集合$X$上的二元关系$\sim$叫作\textbf{等价关系},如果任取$x,x',x''\in X$,满足:
	\begin{enumerate}
		\item 反身性:$x\sim x$;
		\item 对称性:$x\sim x'\implies x'\sim x$;
		\item 传递性:$x\sim x'$且$x'\sim x''\implies x\sim x''$.
	\end{enumerate}
\end{newdef}


元素$a,b\in X$不具有等价关系记作$a\not\sim b$.

\begin{newdef}[等价类]
	在集合$X$中,与给定元素$x$等价的所有元素的集合,叫作包含$x$的\textbf{等价类},记为
	\[\overline{x}:=\{x'\in X|x'\sim x\}\subset X.\]
	任意元素$x'\in\overline{x}$叫作$\overline{x}$的\textbf{代表元}.
\end{newdef}


我们将在下述两个对偶的命题中看到,等价关系与集合的分类有着密切的联系.

先叙述一下记号. 集合$X$的所有子集组成的集合称为$X$的\textbf{幂集},记作
\[\mathcal{P}(X):=\bigcup_{S\subset X} S.\]
如果集合$X$能够表示成其若干非空子集的不交并,那么这些称这些子集的集合为$X$的一个\textbf{划分},并记作$\pi(X)$.


\begin{newprop}
	由关系$\sim$确定的所有等价类的集合是集合$X$的一个划分,即$X$是这些等价类的不交并,记作
	\[\pi_{\sim}(X)=\{\overline{x}\in \mathcal{P}(X)|x\in X\}.\]
\end{newprop}


\begin{newproof}
	注意到$x\in \overline{x}$,因此$\bigcup_{x\in X}\overline{x}=X$. 如果存在两个不同的等价类$\overline{x_1},\overline{x_2}$,使得$\overline{x_1}\bigcap\overline{x_2}=x'$,那么有$x'\sim x_1$和$x'\sim x_2$. 由等价关系的传递性知$x_1\sim x_2$,即$\overline{x_1}=\overline{x_2}$,矛盾!故对任意两个不同的等价类$\overline{x_1},\overline{x_2}$,都有$\overline{x_1}\bigcap\overline{x_2}=\varnothing.$
\end{newproof}


\begin{newprop}
	如果$\pi(X)$是将集合$X$分成不相交子集$C_t$的一个划分,则$C_t$是由某一等价关系$\sim$确定的等价类.
\end{newprop}


\begin{newproof}
	设集合$X=\bigcup_{C_t\in\pi(X)}C_t$. 根据划分的定义,每个元素$x\in X$仅被包含在一个子集$C_a$中. 定义$x\sim x'$当且仅当$x$与$x'$属于同一个集合$C_a$. 容易验证这个关系是反身、对称且传递的,即$\sim$是一个等价关系. 进一步根据等价类的定义,若$x\in C_a$,则$C_a$就是等价类$\overline{x}$. 所以对于我们定义的这种等价关系,有$\pi(X)=\pi_{\sim}(X)$.
\end{newproof}

~\\
由于等价关系与集合的划分是一一对应的,因此对应于等价关系$\sim$的划分$\pi_{\sim}(X)$通常记作$X/\sim$,也叫作$X$关于$\sim$的\textbf{商集}. 定义满射
$p:x\longmapsto\overline{x}$,并称之为$X$到商集$X/\sim$上的\textbf{自然投影}(natural projection).

方便起见,我们引入交换图(commutative diagram)作为工具. 例如下图:
\[\xymatrix{
	A\ar[r]^{f}\ar[d]_{g} & B\ar[d]^{\varphi} \\
	C\ar[r]_{\psi}        & D 
}\]
我们称这个图表交换,当且仅当$\varphi\circ f = \psi\circ g$,即A中的元沿着两条路到达D得到同一个元.


\begin{newprop}
	给定映射$f:X\longrightarrow Y$. 在$X$上定义等价关系如下
	\[a\sim b \iff f(a) = f(b).\]
	设$p:X/\sim\longrightarrow Y$为自然投影. 则存在唯一映射$\overline{f}$,使得下图交换
    \[\xymatrix{
    	X\ar[r]^{p}\ar[dr]_{f} & X/\sim\ar[d]^{\overline{f}} \\
    	  & Y 
    }\]
    并且$\overline{f}$是单射.
\end{newprop}


\newpage
\begin{newproof}
	令$\overline{f}(\overline{x})=f(x)$,则有
	\[
	\overline{f}\circ p(x)=\overline{f}(p(x))=\overline{f}(\overline{x})=f(x).
	\]
	这就证明了$\overline{f}$的存在性. 若$\overline{\phi}$满足$\overline{\phi}\circ p=f$,则对任意$\overline{x}\in X/\sim$, 有
	\[
	\overline{\phi}(\overline{x})=\overline{\phi}(p(x))=\overline{\phi}\circ p(x)=f(x)=\overline{f}(\overline{x}),
	\]
	即$\overline{\phi}=\overline{f}$,因此$\overline{f}$是唯一的. $\overline{f}$是单射由下述事实给出:$\;\forall\;\overline{x_1},\overline{x_2}\in X/\sim$,
	\[
	\overline{f}(\overline{x_1})=\overline{f}(\overline{x_2})\iff
	f(x_1)=f(x_2)\iff \overline{x_1}=\overline{x_2}.
	\] 
\end{newproof}
现在我们知道,这个交换图直观地描述了一个分解式
\begin{equation}
	f=\overline{f}\circ p,
\end{equation}
映射$f$总可以写成一个满射$p:x\longmapsto\overline{x}$和一个单射$\overline{f}:\overline{x}\longmapsto f(x)$的乘积.

\bibliographystyle{ieeetr}
\bibliography{reference}
\addcontentsline{toc}{chapter}{参考文献} 

\end{document}
